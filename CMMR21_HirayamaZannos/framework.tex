
%%%%%%%%%%%%%%%%%%%% file cmmr2021_template.tex %%%%%%%%%%%%%%%%%%%%%
%
% This is the LaTeX source for the instructions to authors using
% the LaTeX document class 'llncs.cls' for contributions to
% the Lecture Notes in Computer Sciences series.
% http://www.springer.com/lncs       Springer Heidelberg 2006/05/04
%
% It may be used as a template for your own input - copy it
% to a new file with a new name and use it as the basis
% for your article.
%
% NB: the document class 'llncs' has its own and detailed documentation, see
% ftp://ftp.springer.de/data/pubftp/pub/tex/latex/llncs/latex2e/llncsdoc.pdf
%
%%%%%%%%%%%%%%%%%%%%%%%%%%%%%%%%%%%%%%%%%%%%%%%%%%%%%%%%%%%%%%%%%%%


\documentclass[runningheads,a4paper]{llncs}
\usepackage{times}
\usepackage{amssymb}
\setcounter{tocdepth}{3}
\usepackage{graphicx}
\usepackage{url}
\newcommand{\keywords}[1]{\par\addvspace\baselineskip
\noindent\keywordname\enspace\ignorespaces#1}

\pagestyle{headings}

\begin{document}

\mainmatter  % start of an individual contribution

% first the title is needed
\title{Towards an Aesthetic of Hybrid Performance Practice: Incorporating Motion Tracking, Gestural and Telematic Techniques in Audiovisual Performance}

% a short form should be given in case it is too long for the running head
\titlerunning{Aesthetic of Hybrid Performance Practice}

% the name(s) of the author(s) follow(s) next
%
% NB: Chinese authors should write their first names(s) in front of
% their surnames. This ensures that the names appear correctly in
% the running heads and the author index.
%
\author{Haruka Hirayama\inst{1}\and Ioannis Zannos\inst{2}}
%
% if the names of the authors are too long for the running head, please use the format: AuthorA et al.
% \authorrunning{AuthorA and AuthorB (or AuthorA et al. if too long)}

% the affiliations are given next; don't give your e-mail address
% unless you accept that it will be published
\institute{Hokkaido Information University, Dept. of Information Media \and Ionian University, Dept. of Audiovisual Arts\\ \email{hiraryama@do-johodai.ac.jp}\\ \email{zannos@gmail.com}}

%
% NB: a more complex sample for affiliations and the mapping to the
% corresponding authors can be found in the file "llncs.dem"
% (search for the string "\mainmatter" where a contribution starts).
% "llncs.dem" accompanies the document class "llncs.cls".
%

\maketitle

\begin{abstract}
This paper discusses composing works for interactive live performance
based on the comparison between recent artworks and experimental work
methods of the two authors. We compare the different interaction design
strategies employed and discuss the factors which influenced the choice
of methods for motion tracking and their influence on body movements
when coupled with the generation of sound during the performance. We
consider the resulting artworks as hybrid artforms that combine aspects
of music composition, improvised sound performance, and stage
performance or dance. A high-level comparison of the technical and
practical aspects of said works is provided. It can be argued that the
new expressive potential and the wealth of possibilities to be explored
warrants further work in this direction, and that systematic comparison
of the interactive characteristics and expressive affordances of the
systems developed are useful in guiding further research in the
development of novel hybrid performance forms.

\keywords{Interactive Music Performance, Audio-Visual Art, Gesture
Mapping, Telematic Art, Embodied Performance}
\end{abstract}

\noindent
\section{CarnaticMusicTheory}
\label{sec:orga13830d}
\begin{itemize}
\item \url{https://www.youtube.com/watch?v=ZnObqYfb-0M}
\item \url{https://www.youtube.com/watch?v=ESCbw9Ph9m4\&list=RDZnObqYfb-0M\&index=13}
\item \url{https://www.youtube.com/watch?v=3D6RjQAR1yU}
\item \url{https://www.youtube.com/watch?v=QnTYAZXBrJ8}
\end{itemize}

\subsection{Swara}
\label{sec:org2a8b2cf}
\begin{itemize}
\item \url{https://www.youtube.com/watch?v=EL4OEaYAdcw}
\item \url{https://www.youtube.com/watch?v=H7u6Ik7sMxM}
\item \url{https://www.youtube.com/watch?v=107FA-Ps0OU}
\end{itemize}

\subsection{Shruti}
\label{sec:orge097de9}
\begin{itemize}
\item \url{https://www.youtube.com/watch?v=pzz7NRhOv8Y}
\item \url{https://www.youtube.com/watch?v=ipYLnhC5YDo}
\item \url{https://www.youtube.com/watch?v=YfF8ZEVOYks}
\item \url{https://www.youtube.com/watch?v=ouKyxvGvgGY}
\end{itemize}

\subsection{GatiAndTisram}
\label{sec:org9337233}
\begin{itemize}
\item \url{https://www.youtube.com/watch?v=YW2B\_eHyhpY}
\item \url{https://www.youtube.com/watch?v=yKmJ552fFBw}
\item \url{https://www.youtube.com/watch?v=ON1bwy\_Xff4}
\item \url{https://www.youtube.com/watch?v=wlQTf3RDQag}
\item \url{https://www.youtube.com/watch?v=D4d88IgvhK4}
\item \url{https://www.youtube.com/watch?v=gd9JiwfkYg4}
\item \url{https://www.youtube.com/watch?v=9lg1PL98tZI}
\end{itemize}

\subsection{Talas}
\label{sec:orgfdae831}
\begin{itemize}
\item \url{https://www.youtube.com/watch?v=XcMqTycAizA}
\item \url{https://www.youtube.com/watch?v=nbKOT4mVxpw}
\item \url{https://www.youtube.com/watch?v=H17Uk\_v4Sbw\&list=PLDJF96khF2cAm8kcu6Msmrl0z83enO7kx}
\item \url{https://www.youtube.com/watch?v=R7d4T5L6qKI}
\item \url{https://www.youtube.com/watch?v=R7d4T5L6qKI}
\item \url{https://www.youtube.com/watch?v=ZYhK2iAU9QI}
\end{itemize}

\subsection{Ragas}
\label{sec:org471f93a}
\begin{itemize}
\item \url{https://www.youtube.com/watch?v=Vt4FwN8vWag}
\item \url{https://www.youtube.com/watch?v=geSOK65rhdQ}
\item \url{https://www.youtube.com/watch?v=7aABhRKRhDI}
\item \url{https://www.youtube.com/watch?v=OwPXH4gZitw}
\end{itemize}

\subsection{Gamaka}
\label{sec:org6a76645}
\begin{itemize}
\item \url{https://www.youtube.com/watch?v=AIPraIlSmIk\&list=RDZnObqYfb-0M\&index=12}
\item \url{https://www.youtube.com/watch?v=N7oz3v2ePWo}
\item \url{https://www.youtube.com/watch?v=-2r5unvlKWw}
\item \url{https://www.youtube.com/watch?v=Vc5fOilytl4}
\item \url{https://www.youtube.com/watch?v=eJZ4J3vu-tc}
\end{itemize}

\subsection{Brigha}
\label{sec:org86886e5}
\begin{itemize}
\item \url{https://www.youtube.com/watch?v=8i1VyR1zJ48}
\item \url{https://www.youtube.com/watch?v=GnI4R9Bdl3g}
\item \url{https://www.youtube.com/watch?v=MCqFbt8nWZ8}
\end{itemize}

\subsection{kan swar}
\label{sec:org2c8cdb3}
\begin{itemize}
\item \url{https://www.youtube.com/watch?v=s44soTmGuUA}
\end{itemize}

\subsection{Symmetry}
\label{sec:org0778a34}
\begin{itemize}
\item \url{https://www.youtube.com/watch?v=hOuu4NTZqP8\&list=RDZnObqYfb-0M\&index=16}
\end{itemize}

\subsection{Mora}
\label{sec:org538ab66}
\begin{itemize}
\item \url{https://www.jstor.org/stable/852807?seq=1}
\end{itemize}

\subsection{Kriti}
\label{sec:org2ed1911}
\begin{itemize}
\item \url{https://academic.oup.com/mts/article/23/1/74/995564}
\item \url{https://www.youtube.com/watch?v=fnHHE39VGz4}
\item \url{https://www.youtube.com/watch?v=hZuePOXXhyI\&list=PL8dh-Wedpdiey1jlMXan6m2ONMR1HmAKb}
\url{http://www.youtube.com/watch?v=\_aOQd9UuaOQ}
\url{http://www.youtube.com/watch?v=dRWi2gVN2kM}
\url{http://www.youtube.com/watch?v=IOaruZrl0c4}
\item \url{https://www.youtube.com/watch?v=Joyk\_EMtzn0}
\end{itemize}

\section{CarnaticPercussion}
\label{sec:orgf765fdb}
\begin{itemize}
\item \url{https://www.youtube.com/watch?v=5FqfplZcvys}
\item \url{https://www.youtube.com/watch?v=Cs6IWz5BgOs}
\item \url{https://www.youtube.com/watch?v=bBKAEwYzk\_o}
\item \url{https://www.youtube.com/watch?v=M4q3G5-d7x4}
\item \url{https://www.youtube.com/watch?v=U4BcxbYKvKg}
\item \url{https://www.youtube.com/watch?v=lrGgllzIgic}
\item \url{https://www.youtube.com/watch?v=pLTwkTMZaFI}
\item \url{https://www.youtube.com/watch?v=Z6fTb0Z3m54}
\item \url{https://www.youtube.com/watch?v=55-GBXNtpE4}
\end{itemize}

\subsection{Konnakol}
\label{sec:org1f5ab6b}

\subsubsection{Other}
\label{sec:org4647110}
\begin{itemize}
\item \url{https://www.youtube.com/watch?v=nE0fpD5MMAs}
\item \url{https://www.youtube.com/watch?v=9mozmHgg9Sk}
\item \url{https://www.youtube.com/watch?v=9k8icmmwHSY}
\item \url{https://www.youtube.com/watch?v=KsvKQhOeQjQ}
\item \url{https://www.youtube.com/watch?v=RLTDfoneAJ0}
\item \url{https://www.youtube.com/watch?v=21eS3GXS47k}
\item \url{https://www.youtube.com/watch?v=wlY7rp9xm0I}
\item \url{https://www.youtube.com/watch?v=76kuNLs61xI}
\item \url{https://www.youtube.com/watch?v=OyyfLtYQcwI}
\item \url{https://www.youtube.com/watch?v=qdASeGlQW1g}
\item \url{https://www.youtube.com/watch?v=ernL2Q9FsJw}
\item \url{https://www.youtube.com/watch?v=ve98rXnpg\_Y}
\item \url{https://www.youtube.com/watch?v=YcvLr39v0sY}
\end{itemize}

\subsubsection{Manjunath}
\label{sec:org9b12d33}
\begin{itemize}
\item \url{https://www.youtube.com/watch?v=mOMLRMfIYf0}
\item \url{https://www.youtube.com/watch?v=h6VS7KlZeNQ}
\item \url{https://www.youtube.com/watch?v=SRCufQbPm9w}
\item \url{https://www.youtube.com/watch?v=l5DArpqLP28}
\item \url{https://www.youtube.com/watch?v=83jyFoGjK\_g}
\item \url{https://www.youtube.com/watch?v=nTSfMAQyhIA}
\item \url{https://www.youtube.com/watch?v=GOKiCedfoOo}
\item \url{https://www.youtube.com/watch?v=7GglM5y9Ju0}
\item \url{https://www.youtube.com/watch?v=a7xQFHUIQoA}
\item \url{https://www.youtube.com/watch?v=18HL4dd-Xig}
\item \url{https://www.youtube.com/watch?v=lRcne9GaKtY}
\item \url{https://www.youtube.com/watch?v=-mS06lEmY3s}
\item \url{https://www.youtube.com/watch?v=e-7SGB0RKjE}
\item \url{https://www.youtube.com/watch?v=lhAxN7hGIR8}
\item \url{https://www.youtube.com/watch?v=WCfEL2SFOao}
\item \url{https://www.youtube.com/watch?v=7DEADUBo-x8}
\item \url{https://www.youtube.com/watch?v=Yrm0P4OLuM8}
\item \url{https://www.youtube.com/watch?v=Y5rgIrkHwyg}
\item \url{https://www.youtube.com/watch?v=SoPjy6kpi1A}
\item \url{https://www.youtube.com/watch?v=hmY1hEjK2h0}
\item \url{https://www.youtube.com/watch?v=LlzSl52zHMA}
\item \url{https://www.youtube.com/watch?v=bqMjS64dcD8}
\item \url{https://www.youtube.com/watch?v=6aHWJKJe9mU}
\item \url{https://www.youtube.com/watch?v=Cx4V\_8y7uNM}
\item \url{https://www.youtube.com/watch?v=83jyFoGjK\_g}
\item \url{https://www.youtube.com/watch?v=T6Nm9hZLrLc}
\item \url{https://www.youtube.com/watch?v=TQmMTNnRX6k}
\item \url{https://www.youtube.com/watch?v=Ya1qCq7kk4Y}
\item \url{https://www.youtube.com/watch?v=NXikDhuZH7Y}
\item \url{https://www.youtube.com/watch?v=lhAxN7hGIR8}
\item \url{https://www.youtube.com/watch?v=iPzq0s4\_wl0}
\end{itemize}

\subsubsection{ShivapriyaSomashekar}
\label{sec:org1fe1221}
\begin{itemize}
\item \url{https://www.youtube.com/watch?v=iurhjlBum0o}
\item \url{https://www.youtube.com/watch?v=QNBQxUTTA4s}
\item \url{https://www.youtube.com/watch?v=jA\_3g8zgMf0}
\item \url{https://www.youtube.com/watch?v=LcMO785LNjg}
\item \url{https://www.youtube.com/watch?v=g2ozpJYRw4k}
\item \url{https://www.youtube.com/watch?v=9mfKdlL9Fxo}
\end{itemize}

\subsubsection{JoisSomshekar}
\label{sec:orgeffc68c}
\begin{itemize}
\item \url{https://www.youtube.com/watch?v=YcvLr39v0sY}
\item \url{https://www.youtube.com/watch?v=GA575BJ2HUY}
\end{itemize}

\subsubsection{Shivapriya}
\label{sec:org079288c}
\begin{itemize}
\item \url{https://www.youtube.com/watch?v=cDG3XVsEhwk}
\item \url{https://www.youtube.com/watch?v=rceY1wWi1uM}
\item \url{https://www.youtube.com/watch?v=YhEGlFXp830}
\item \url{https://www.youtube.com/watch?v=sw2PW5\_CoNg}
\end{itemize}

\subsubsection{IndianMathIn20thCtrMinimalism}
\label{sec:orged88953}
\begin{itemize}
\item \url{https://www.youtube.com/watch?v=WmUpQpda6cw}
\item \url{https://www.youtube.com/watch?v=6yXQinqLmqc}
\item \url{https://www.youtube.com/watch?v=y--yvG\_IUoI}
\end{itemize}

\section{Singers}
\label{sec:org0340d4a}

\subsection{Venugopal}
\label{sec:org64e174f}
\begin{itemize}
\item \url{https://www.youtube.com/watch?v=AAfE00Gn00I}
\item \url{https://www.youtube.com/watch?v=fkCpFN9cVPY}
\item \url{https://www.youtube.com/watch?v=t1ek4siKqn4}
\item \url{https://www.youtube.com/watch?v=gpfg-Yyd5PI}
\end{itemize}

\subsection{Chakraborty}
\label{sec:org6a4f2b6}
\begin{itemize}
\item \url{https://www.youtube.com/watch?v=uEqYzdz3Zvg}
\item \url{https://www.youtube.com/watch?v=94pgVJ32D9U}
\item \url{https://www.youtube.com/watch?v=YHdR2A\_1DCg}
\end{itemize}

\subsection{NinaBurmi}
\label{sec:org4dfb1a9}
\begin{itemize}
\item \url{https://www.youtube.com/watch?v=kfBvz2rG-NI}
\item \url{https://www.youtube.com/watch?v=VMJ7xQhJ0n0}
\item \url{https://www.youtube.com/watch?v=zGilSftMcI0}
\item \url{https://www.youtube.com/watch?v=fUKKYizf\_-k}
\end{itemize}

\subsection{ArunaSairam}
\label{sec:org6e92673}
\begin{itemize}
\item \url{https://www.youtube.com/watch?v=c9Cbhpd2zYw}
\item \url{https://www.youtube.com/watch?v=spRQEectgB8}
\item \url{https://www.youtube.com/watch?v=\_K-e0Io3yJk}
\item \url{https://www.youtube.com/watch?v=2jTj9Vo7lio}
\item \url{https://www.youtube.com/watch?v=jQqtGzdteQ8}
\item \url{https://www.youtube.com/watch?v=G2LfJLDinqc}
\item \url{https://www.youtube.com/watch?v=zBAZzPZE5Pk}
\end{itemize}

\subsection{Vaidyanathan}
\label{sec:org94ad462}
\begin{itemize}
\item \url{https://www.youtube.com/watch?v=ks8ugJW4CqI}
\item \url{https://www.youtube.com/watch?v=dXxjnASv1ow}
\item \url{https://www.youtube.com/watch?v=fbyIRRwDOlU}
\end{itemize}

\subsection{ParveenSultana}
\label{sec:org8edc196}
\begin{itemize}
\item \url{https://www.youtube.com/watch?v=PzCZomuHVVQ}
\item \url{https://www.youtube.com/watch?v=9X3vjQXx7xw}
\item \url{https://www.youtube.com/watch?v=NgXRhF9LyrE}
\item \url{https://www.youtube.com/watch?v=Y4x6T4boG8o}
\end{itemize}

\subsection{ShubhaMudgal}
\label{sec:org3ca3d5e}
\begin{itemize}
\item \url{https://www.youtube.com/watch?v=Yh8QfWlSv9Q}
\end{itemize}

\subsection{PrabhaAtre}
\label{sec:orgd985440}
\begin{itemize}
\item \url{https://www.youtube.com/watch?v=sRNg-v1Dg\_4}
\end{itemize}

\subsection{RanjaniAndGayatri}
\label{sec:org6c57126}
\begin{itemize}
\item \url{https://www.youtube.com/watch?v=CoyoCFYMQsc}
\item \url{https://www.youtube.com/watch?v=beJXJVwD3v4}
\end{itemize}

\section{Veena}
\label{sec:orga74f21a}
\begin{itemize}
\item \url{https://www.youtube.com/watch?v=jM9b2Qo5qwM}
\item \url{https://www.youtube.com/watch?v=jM9b2Qo5qwM}
\item \url{https://www.youtube.com/watch?v=zBAZzPZE5Pk}
\item \url{https://www.youtube.com/watch?v=o6M\_kXzdDzI}
\end{itemize}

\section{Panini}
\label{sec:org118ac0e}
\begin{itemize}
\item \url{https://en.wikipedia.org/wiki/Hindu\%E2\%80\%93Arabic\_numeral\_system}
\item \url{https://ashtadhyayi.com/sutraani/sk3183}
\item \url{https://ashtadhyayi.com/sutraani/8/4/67}
\item \url{https://www.youtube.com/watch?v=l3Wo5MYljzc}
\item \url{https://www.youtube.com/watch?v=0emIewicwl0}
\end{itemize}




\section{Gayatri Chakravorty Spivak}
\label{sec:org8fe61bd}
\begin{itemize}
\item \url{https://www.youtube.com/watch?v=SG0bXHVr3mY}
\item \url{https://www.youtube.com/watch?v=garPdV7U3fQ}
\item \url{https://www.youtube.com/watch?v=n8iPj6qka3o}
\item \url{https://www.youtube.com/watch?v=SG0bXHVr3mY}
\item \url{https://www.youtube.com/watch?v=2ZHH4ALRFHw}
\item \url{https://www.youtube.com/watch?v=YBzCwzvudv0}
\end{itemize}


\begin{thebibliography}{4}

\bibitem{ohia} Collins, Karen, Bill Kapralos, and Holly Tessler, eds. The Oxford Handbook of Interactive Audio. Oxford University Press, Oxford (2014).

\bibitem{pid1} People in the Dunes project page,  \url{https://www.facebook.com/peopleinthedunes/}, last accessed 2021/06/15

\bibitem{pid2} People in the Dunes II videos, \url{https://vimeo.com/user88406194}, last accessed 2021/06/15

\bibitem{ide} IDE-Fantasy videos, \url{https://ide-fantasy.tumblr.com/}, last accessed 2021/06/15
 
\bibitem{ide} IDE-Fantasy videos, \url{https://ide-fantasy.tumblr.com/}, last accessed 2021/06/15

\bibitem{lac19} Zannos, I.: SC-Hacks: A Live Coding Framework for Gestural Performance and Electronic Music. In: Proceedings of the 2019 Linux Audio Conference. pp. 121-128. CCRMA, Stanford University (2019).

\bibitem{skin} Kobayashi, Y., Matsuura, H., eds. The Discourse of Repr\'esentation Skinship: The Rhetoric of the Body. University of Tokyo press, Tokyo (2000).
\end{thebibliography}

\end{document}
